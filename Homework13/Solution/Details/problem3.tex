某工厂生产$P_1,P_2,P_3,P_4$四种产品。各产品单位利润分别是100,20,80,140元。
设生产这四种产品需要六种原材料$M_1,M_2,\ldots,M_6$。下列矩阵$A=(a_{ij})_{4\times 6}$,
其中$a_{ij}$是$P_i$产品需要原材料$M_j$的量。假定现有$M_1,M_2,\ldots,M_6$的数量
分别是1000,30,20,20,5,195单位。试问如何安排生产,使利润达到最大。
\begin{align*}
    A = \left(
        \begin{matrix}
            10 & 1 & 0 & 0 & 0 & 2\\
            15 & 0 & 1 & 0 & 0 & 3\\
            18 & 0 & 0 & 1 & 0 & 4\\
            21 & 0 & 0 & 0 & 1 & 5\\
        \end{matrix}
    \right)
\end{align*}
