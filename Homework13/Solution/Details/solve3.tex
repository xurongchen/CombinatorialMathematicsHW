\begin{solution}
    \begin{align*}
        &\max z = 100x_1 + 20x_2 + 80x_3 + 140x_4\\
        &s.t. \begin{cases}
            10x_1 + 15x_2 + 18x_3 + 21x_4 + x_5 = 1000\\
            x_1 + x_6 = 30\\
            x_2 + x_7 = 20\\
            x_3 + x_8 = 20\\
            x_4 + x_9 = 5\\
            2x_1 + 3x_2 + 4x_3 + 5x_4 + x_{10} = 195\\
            x_i \ge 0, i \in \{1,2,\ldots, 10\}
       \end{cases}\\
     \end{align*}
     构建初始单纯形表如表\ref{te1}。
      \begin{table}[!h]
         \centering
         \caption{单纯形表E1}
         \label{te1}
         \begin{tabular}{c|ccccccccccc} 
         \toprule
                  &$x_1$  &$x_2$  &$x_3$  &$x_4$  &$x_5$  &$x_6$  &$x_7$  &$x_8$  &$x_9$  &$x_{10}$ &$b$    \\\hline
         $x_5$    &$10$   &$15$   &$18$   &$21$   &$1$    &$0$    &$0$    &$0$    &$0$    &$0$      &$1000$ \\
         $x_6$    &$1$    &$0$    &$0$    &$0$    &$0$    &$1$    &$0$    &$0$    &$0$    &$0$      &$30$   \\
         $x_7$    &$0$    &$1$    &$0$    &$0$    &$0$    &$0$    &$1$    &$0$    &$0$    &$0$      &$20$   \\
         $x_8$    &$0$    &$0$    &$1$    &$0$    &$0$    &$0$    &$0$    &$1$    &$0$    &$0$      &$20$   \\
         $x_9$    &$0$    &$0$    &$0$    &$1$    &$0$    &$0$    &$0$    &$0$    &$1$    &$0$      &$5$    \\
         $x_{10}$ &$2$    &$3$    &$4$    &$5$    &$0$    &$0$    &$0$    &$0$    &$0$    &$1$      &$195$  \\
         $z$      &$100$  &$20$   &$80$   &$140$  &$0$    &$0$    &$0$    &$0$    &$0$    &$0$      &$0$    \\
         \bottomrule
         \end{tabular}
      \end{table}

      从$z$所在行的正数中最大的一个所对应的变量作为基变量,这里选择$x_4$。
      由于拿$x_4$所在列的正系数去除$b$所在列的数的结果$\frac{1000}{21}>\frac{195}{5}>\frac{5}{1}$,
      故取$x_{9}$离开基变量。

      然后对该矩阵进行行变换,使$x_4$所在列变成单位向量,如表\ref{te2}。

      \begin{table}[!h]
        \centering
        \caption{单纯形表E2}
        \label{te2}
        \begin{tabular}{c|ccccccccccc} 
        \toprule
                 &$x_1$  &$x_2$  &$x_3$  &$x_4$  &$x_5$  &$x_6$  &$x_7$  &$x_8$  &$x_9$  &$x_{10}$ &$b$    \\\hline
        $x_4$    &$0$    &$0$    &$0$    &$1$    &$0$    &$0$    &$0$    &$0$    &$1$    &$0$      &$5$    \\
        $x_5$    &$10$   &$15$   &$18$   &$0$    &$1$    &$0$    &$0$    &$0$    &$-21$  &$0$      &$895$  \\
        $x_6$    &$1$    &$0$    &$0$    &$0$    &$0$    &$1$    &$0$    &$0$    &$0$    &$0$      &$30$   \\
        $x_7$    &$0$    &$1$    &$0$    &$0$    &$0$    &$0$    &$1$    &$0$    &$0$    &$0$      &$20$   \\
        $x_8$    &$0$    &$0$    &$1$    &$0$    &$0$    &$0$    &$0$    &$1$    &$0$    &$0$      &$20$   \\
        $x_{10}$ &$2$    &$3$    &$4$    &$0$    &$0$    &$0$    &$0$    &$0$    &$-5$   &$1$      &$170$  \\
        $z$      &$100$  &$20$   &$80$   &$0$    &$0$    &$0$    &$0$    &$0$    &$-140$ &$0$      &$-700$ \\
        \bottomrule
        \end{tabular}
     \end{table}

      从$z$所在行的正数中最大的一个所对应的变量作为基变量,这里选择$x_1$。
      由于拿$x_1$所在列的正系数去除$b$所在列的数的结果$\frac{895}{10}>\frac{170}{2}>\frac{30}{1}$,
      故取$x_{6}$离开基变量。

     然后对该矩阵进行行变换,使$x_1$所在列变成单位向量,如表\ref{te3}。

     \begin{table}[!h]
        \centering
        \caption{单纯形表E3}
        \label{te3}
        \begin{tabular}{c|ccccccccccc} 
        \toprule
                 &$x_1$  &$x_2$  &$x_3$  &$x_4$  &$x_5$  &$x_6$  &$x_7$  &$x_8$  &$x_9$  &$x_{10}$ &$b$    \\\hline
        $x_1$    &$1$    &$0$    &$0$    &$0$    &$0$    &$1$    &$0$    &$0$    &$0$    &$0$      &$30$   \\
        $x_4$    &$0$    &$0$    &$0$    &$1$    &$0$    &$0$    &$0$    &$0$    &$1$    &$0$      &$5$    \\
        $x_5$    &$0$    &$15$   &$18$   &$0$    &$1$    &$-10$  &$0$    &$0$    &$-21$  &$0$      &$595$  \\
        $x_7$    &$0$    &$1$    &$0$    &$0$    &$0$    &$0$    &$1$    &$0$    &$0$    &$0$      &$20$   \\
        $x_8$    &$0$    &$0$    &$1$    &$0$    &$0$    &$0$    &$0$    &$1$    &$0$    &$0$      &$20$   \\
        $x_{10}$ &$0$    &$3$    &$4$    &$0$    &$0$    &$-2$   &$0$    &$0$    &$-5$   &$1$      &$110$  \\
        $z$      &$0$    &$20$   &$80$   &$0$    &$0$    &$-100$ &$0$    &$0$    &$-140$ &$0$      &$-3700$ \\
        \bottomrule
        \end{tabular}
     \end{table}

     从$z$所在行的正数中最大的一个所对应的变量作为基变量,这里选择$x_3$。
     由于拿$x_3$所在列的正系数去除$b$所在列的数的结果$\frac{595}{18}>\frac{110}{4}>\frac{20}{1}$,
     故取$x_{8}$离开基变量。

    然后对该矩阵进行行变换,使$x_3$所在列变成单位向量,如表\ref{te4}。

    \begin{table}[!h]
        \centering
        \caption{单纯形表E4}
        \label{te4}
        \begin{tabular}{c|ccccccccccc} 
        \toprule
                 &$x_1$  &$x_2$  &$x_3$  &$x_4$  &$x_5$  &$x_6$  &$x_7$  &$x_8$  &$x_9$  &$x_{10}$ &$b$    \\\hline
        $x_1$    &$1$    &$0$    &$0$    &$0$    &$0$    &$1$    &$0$    &$0$    &$0$    &$0$      &$30$   \\
        $x_3$    &$0$    &$0$    &$1$    &$0$    &$0$    &$0$    &$0$    &$1$    &$0$    &$0$      &$20$   \\
        $x_4$    &$0$    &$0$    &$0$    &$1$    &$0$    &$0$    &$0$    &$0$    &$1$    &$0$      &$5$    \\
        $x_5$    &$0$    &$15$   &$0$    &$0$    &$1$    &$-10$  &$0$    &$-18$  &$-21$  &$0$      &$235$  \\
        $x_7$    &$0$    &$1$    &$0$    &$0$    &$0$    &$0$    &$1$    &$0$    &$0$    &$0$      &$20$   \\
        $x_{10}$ &$0$    &$3$    &$0$    &$0$    &$0$    &$-2$   &$0$    &$-4$   &$-5$   &$1$      &$30$   \\
        $z$      &$0$    &$20$   &$0$    &$0$    &$0$    &$-100$ &$0$    &$-80$  &$-140$ &$0$      &$-5300$ \\
        \bottomrule
        \end{tabular}
     \end{table}

     从$z$所在行的正数中最大的一个所对应的变量作为基变量,这里选择$x_2$。
     由于拿$x_3$所在列的正系数去除$b$所在列的数的结果$\frac{20}{1}>\frac{235}{15}>\frac{30}{3}$,
     故取$x_{10}$离开基变量。

    然后对该矩阵进行行变换,使$x_2$所在列变成单位向量,如表\ref{te5}。

    \begin{table}[!h]
        \centering
        \caption{单纯形表E5}
        \label{te5}
        \begin{tabular}{c|ccccccccccc} 
        \toprule
                 &$x_1$  &$x_2$  &$x_3$  &$x_4$  &$x_5$  &$x_6$  &$x_7$  &$x_8$  &$x_9$  &$x_{10}$ &$b$    \\\hline
        $x_1$    &$1$    &$0$    &$0$    &$0$    &$0$    &$1$    &$0$    &$0$    &$0$    &$0$      &$30$   \\
        $x_{2}$  &$0$    &$1$    &$0$    &$0$    &$0$    &$-\frac{2}{3}$   &$0$    &$-\frac{4}{3}$   &$-\frac{5}{3}$   &$\frac{1}{3}$      &$10$   \\
        $x_3$    &$0$    &$0$    &$1$    &$0$    &$0$    &$0$    &$0$    &$1$    &$0$    &$0$      &$20$   \\
        $x_4$    &$0$    &$0$    &$0$    &$1$    &$0$    &$0$    &$0$    &$0$    &$1$    &$0$      &$5$    \\
        $x_5$    &$0$    &$0$    &$0$    &$0$    &$1$    &$0$    &$0$    &$2$    &$4$    &$-5$     &$85$   \\
        $x_7$    &$0$    &$0$    &$0$    &$0$    &$0$    &$\frac{2}{3}$    &$1$    &$\frac{4}{3}$    &$\frac{5}{3}$    &$-\frac{1}{3}$      &$10$   \\
        $z$      &$0$    &$0$    &$0$    &$0$    &$0$    &$-\frac{260}{3}$ &$0$    &$-\frac{160}{3}$ &$-\frac{320}{3}$ &$\frac{20}{3}$      &$-5500$ \\
        \bottomrule
        \end{tabular}
     \end{table}

     由于$z$所在行的所有数都非正,问题结束。最优解为$x_1=30,x_2=10,x_3=20,x_4=5$,最优值为$z = 100x_1 + 20x_2 + 80x_3 + 140x_4 = 5500$。

    即分别生产$P_1,P_2,P_3,P_4$的量为30,10,20,5单位利润最大。最大利润为5500元。
\end{solution}