\begin{solution}
   \begin{enumerate}
      \item 
      \begin{align*}
         &\max z = 2x_1 + x_2\\
         &s.t. \begin{cases}
            x_1 + 4x_2 + x_3 = 32\\
            x_1 + x_2 + x_4 = 11\\
            5x_1 + x_2 + x_5 = 35\\
            x_i \ge 0, i \in \{1,2,\ldots, 5\}
        \end{cases}\\
      \end{align*}
      构建初始单纯形表如表\ref{ta1}。
      \begin{table}[!h]
         \centering
         \caption{单纯形表A1}
         \label{ta1}
         \begin{tabular}{c|cccccc} 
         \toprule
                &$x_1$  &$x_2$  &$x_3$  &$x_4$  &$x_5$  &$b$   \\\hline
         $x_3$  &$1$    &$4$    &$1$    &$0$    &$0$    &$32$   \\
         $x_4$  &$1$    &$1$    &$0$    &$1$    &$0$    &$11$   \\
         $x_5$  &$5$    &$1$    &$0$    &$0$    &$1$    &$35$   \\
         $z$    &$2$    &$1$    &$0$    &$0$    &$0$    &$0$   \\
         \bottomrule
         \end{tabular}
      \end{table}

      从$z$所在行的正数中最大的一个所对应的变量作为基变量,这里选择$x_1$。
      由于拿$x_1$所在列的正系数去除$b$所在列的数的结果$\frac{32}{1}>\frac{11}{1}>\frac{35}{5}$,
      故取$x_{5}$离开基变量。

      然后对该矩阵进行行变换,使$x_1$所在列变成单位向量,如表\ref{ta2}。
      
      \begin{table}[!h]
         \centering
         \caption{单纯形表A2}
         \label{ta2}
         \begin{tabular}{c|cccccc} 
         \toprule
                &$x_1$  &$x_2$  &$x_3$  &$x_4$  &$x_5$  &$b$   \\\hline
         $x_1$  &$1$    &$\frac{1}{5}$    &$0$    &$0$    &$\frac{1}{5}$    &$7$   \\
         $x_3$  &$0$    &$\frac{19}{5}$    &$1$    &$0$    &$-\frac{1}{5}$    &$25$   \\
         $x_4$  &$0$    &$\frac{4}{5}$     &$0$    &$1$    &$-\frac{1}{5}$    &$4$   \\
         $z$    &$0$    &$\frac{3}{5}$    &$0$    &$0$    &$-\frac{2}{5}$    &$-14$   \\
         \bottomrule
         \end{tabular}
      \end{table}

      从$z$所在行的正数中最大的一个所对应的变量作为基变量,这里选择$x_2$。
      由于拿$x_1$所在列的正系数去除$b$所在列的数的结果$\frac{7}{\frac{1}{5}}>\frac{25}{\frac{19}{5}}>\frac{4}{\frac{4}{5}}$,
      故取$x_{4}$离开基变量。

      然后对该矩阵进行行变换,使$x_2$所在列变成单位向量,如表\ref{ta3}。

      \begin{table}[!h]
         \centering
         \caption{单纯形表A3}
         \label{ta3}
         \begin{tabular}{c|cccccc} 
         \toprule
                &$x_1$  &$x_2$  &$x_3$  &$x_4$  &$x_5$  &$b$   \\\hline
         $x_1$  &$1$    &$0$    &$0$    &$-\frac{1}{4}$    &$\frac{1}{4}$    &$6$   \\
         $x_2$  &$0$    &$1$    &$0$    &$\frac{5}{4}$    &$-\frac{1}{4}$    &$5$   \\
         $x_3$  &$0$    &$0$    &$1$    &$-\frac{19}{4}$    &$\frac{3}{4}$    &$6$   \\
         $z$    &$0$    &$0$    &$0$    &$-\frac{3}{4}$    &$-\frac{1}{4}$    &$-17$   \\
         \bottomrule
         \end{tabular}
      \end{table}

      由于$z$所在行的所有数都非正,问题结束。最优解为$x_1=6,x_2=5$,最优值为$z = 2x_1 + x_2=17$。
      \item 
      \begin{align*}
         &\max z = x_1 + x_2\\
         &s.t. \begin{cases}
            4x_1 + 5x_2 + x_3 = 10\\
            5x_1 + 2x_2 + x_4 = 10\\
            3x_1 + 8x_2 + x_5 = 12\\
            x_i \ge 0, i \in \{1,2,\ldots, 5\}
        \end{cases}\\
      \end{align*}
      构建初始单纯形表如表\ref{tb1}。
      \begin{table}[!h]
         \centering
         \caption{单纯形表B1}
         \label{tb1}
         \begin{tabular}{c|cccccc} 
         \toprule
                &$x_1$  &$x_2$  &$x_3$  &$x_4$  &$x_5$  &$b$   \\\hline
         $x_3$  &$4$    &$5$    &$1$    &$0$    &$0$    &$10$   \\
         $x_4$  &$5$    &$2$    &$0$    &$1$    &$0$    &$10$   \\
         $x_5$  &$3$    &$8$    &$0$    &$0$    &$1$    &$12$   \\
         $z$    &$1$    &$1$    &$0$    &$0$    &$0$    &$0$   \\
         \bottomrule
         \end{tabular}
      \end{table}

      从$z$所在行的正数中最大的一个所对应的变量作为基变量,因为这里两者一样,不妨选为$x_1$。
      由于拿$x_1$所在列的正系数去除$b$所在列的数的结果$\frac{12}{3}>\frac{10}{4}>\frac{10}{5}$,
      故取$x_{4}$离开基变量。

      然后对该矩阵进行行变换,使$x_1$所在列变成单位向量,如表\ref{tb2}。
      
      \begin{table}[!h]
         \centering
         \caption{单纯形表B2}
         \label{tb2}
         \begin{tabular}{c|cccccc} 
         \toprule
                &$x_1$  &$x_2$  &$x_3$  &$x_4$  &$x_5$  &$b$   \\\hline
         $x_1$  &$1$    &$\frac{2}{5}$    &$0$     &$\frac{1}{5}$  &$0$  &$2$   \\
         $x_3$  &$0$    &$\frac{17}{5}$    &$1$    &$-\frac{4}{5}$   &$0$ &$2$   \\
         $x_5$  &$0$    &$\frac{34}{5}$     &$0$   &$-\frac{3}{5}$  &$1$  &$4$   \\
         $z$    &$0$    &$\frac{3}{5}$    &$0$     &$-\frac{1}{5}$  &$0$  &$-2$   \\
         \bottomrule
         \end{tabular}
      \end{table}

      从$z$所在行的正数中最大的一个所对应的变量作为基变量,这里选择$x_2$。
      由于拿$x_1$所在列的正系数去除$b$所在列的数的结果$\frac{25}{\frac{2}{5}}>\frac{7}{\frac{17}{5}}>\frac{4}{\frac{34}{5}}$,
      故取$x_{5}$离开基变量。

      然后对该矩阵进行行变换,使$x_2$所在列变成单位向量,如表\ref{tb3}。

      \begin{table}[!h]
         \centering
         \caption{单纯形表B3}
         \label{tb3}
         \begin{tabular}{c|cccccc} 
         \toprule
                &$x_1$  &$x_2$  &$x_3$  &$x_4$  &$x_5$  &$b$   \\\hline
         $x_1$  &$1$    &$0$    &$0$    &$\frac{4}{17}$    &$-\frac{3}{85}$    &$\frac{30}{17}$   \\
         $x_2$  &$0$    &$1$    &$0$    &$-\frac{3}{34}$    &$\frac{5}{34}$    &$\frac{10}{17}$   \\
         $x_3$  &$0$    &$0$    &$1$    &$-\frac{1}{2}$    &$-\frac{1}{2}$    &$0$   \\
         $z$    &$0$    &$0$    &$0$    &$-\frac{5}{34}$    &$-\frac{3}{34}$    &$-\frac{40}{17}$   \\
         \bottomrule
         \end{tabular}
      \end{table}

      由于$z$所在行的所有数都非正,问题结束。最优解为$x_1=\frac{30}{17},x_2=\frac{10}{17}$,最优值为$z = x_1 + x_2=\frac{40}{17}$。
      \item 
      \begin{align*}
         &\max z = 2x_1 + 2x_2\\
         &s.t. \begin{cases}
            -x_1 + x_2 + x_3 = 1\\
            x_1 + x_2 + x_4 = 3\\
            x_i \ge 0, i \in \{1,2,3,4\}
        \end{cases}\\
      \end{align*}
      构建初始单纯形表如表\ref{tc1}。
      \begin{table}[!h]
         \centering
         \caption{单纯形表C1}
         \label{tc1}
         \begin{tabular}{c|cccccc} 
         \toprule
                &$x_1$  &$x_2$  &$x_3$  &$x_4$  &$b$   \\\hline
         $x_3$  &$-1$   &$1$    &$1$    &$0$    &$1$   \\
         $x_4$  &$1$    &$1$    &$0$    &$1$    &$3$   \\
         $z$    &$2$    &$2$    &$0$    &$0$    &$0$   \\
         \bottomrule
         \end{tabular}
      \end{table}

      从$z$所在行的正数中最大的一个所对应的变量作为基变量,这里选择$x_1$。
      由于拿$x_1$所在列的正系数去除$b$所在列的数的结果只有$\frac{3}{1}$,
      故取$x_{4}$离开基变量。

      然后对该矩阵进行行变换,使$x_1$所在列变成单位向量,如表\ref{tc2}。

      \begin{table}[!h]
         \centering
         \caption{单纯形表C2}
         \label{tc2}
         \begin{tabular}{c|cccccc} 
         \toprule
                &$x_1$  &$x_2$  &$x_3$  &$x_4$  &$b$   \\\hline
         $x_1$  &$1$    &$1$    &$0$    &$1$    &$3$   \\
         $x_3$  &$0$    &$2$    &$1$    &$1$    &$4$   \\
         $z$    &$0$    &$0$    &$0$    &$-2$    &$-6$   \\
         \bottomrule
         \end{tabular}
      \end{table}

      由于$z$所在行的所有数都非正,问题结束。最优解为存在多个,最优值为$z = 2x_1 + 2x_2=6$。

      \item 
      % 由于第二个不等式转化为小于关系时右侧常数为负,故增加一个非基变量
      \begin{align*}
         &\max z = x_1 + x_2\\
         &s.t. \begin{cases}
            -x_1 + 2x_2 + x_3 = 2\\
            -x_1 + x_2 + x_4 = -2\\
            x_i \ge 0, i \in \{1,2,3,4\}
        \end{cases}\\
      \end{align*}
      构建初始单纯形表如表\ref{td1}。
      \begin{table}[!h]
         \centering
         \caption{单纯形表D1}
         \label{td1}
         \begin{tabular}{c|cccccc} 
         \toprule
                &$x_1$  &$x_2$  &$x_3$  &$x_4$  &$b$   \\\hline
         $x_3$  &$-1$   &$2$    &$1$    &$0$    &$2$   \\
         $x_4$  &$-1$   &$1$    &$0$    &$1$    &$-2$   \\
         $z$    &$1$    &$1$    &$0$    &$0$    &$0$   \\
         \bottomrule
         \end{tabular}
      \end{table}

      由于$x_4$的$b_4<0$,且$A_{41}<0$,于是进行$pviot(4,1)$的转轴操作,
      并使得$x_1$所在列变成单位向量。
      得到单纯形表如表\ref{td2}。

      \begin{table}[!h]
         \centering
         \caption{单纯形表D2}
         \label{td2}
         \begin{tabular}{c|cccccc} 
            \toprule
                   &$x_1$  &$x_2$  &$x_3$  &$x_4$  &$b$   \\\hline
            $x_1$  &$1$    &$-1$   &$0$    &$-1$   &$2$   \\
            $x_3$  &$0$    &$1$    &$1$    &$-1$    &$4$   \\
            $z$    &$1$    &$1$    &$0$    &$0$    &$0$   \\
            \bottomrule
            \end{tabular}
      \end{table}

      此时所有$b>0$,初始化完成。

      对于$e=4$,所有的${A}_{d,e}$均非正,于是此时线性规划的最优解是无穷大。
      
   \end{enumerate}
\end{solution}