\begin{solution}
   \begin{enumerate}
      \item 任意两个拉丁方组合的结果:
      \begin{align*}&
         \left[\begin{matrix}
             (1,3)&(3,1)&(2,4)&(4,2)\\
             (3,2)&(1,4)&(4,1)&(2,3)\\
             (2,1)&(4,3)&(1,2)&(3,4)\\
             (4,4)&(2,2)&(3,3)&(1,1)
         \end{matrix}\right],
         \left[\begin{matrix}
             (3,2)&(1,4)&(4,1)&(2,3)\\
             (2,1)&(4,3)&(1,2)&(3,4)\\
             (1,3)&(3,1)&(2,4)&(4,2)\\
             (4,4)&(2,2)&(3,3)&(1,1)
         \end{matrix}\right],\\&
         \left[\begin{matrix}
            (1,2)&(3,4)&(2,1)&(4,3)\\
            (3,1)&(1,3)&(4,2)&(2,4)\\
            (2,3)&(4,1)&(1,4)&(3,2)\\
            (4,4)&(2,2)&(3,3)&(1,1)
        \end{matrix}\right].
     \end{align*}
     每个组合中均不存在相同元素,即两两正交。又n阶拉丁方的正交族元素数
     至多为n-1,所以上述3个4阶的拉丁方构成正交族;
     \item 上述3个5阶拉丁方任意两个组合均不存在相同元素,即两两正交。
     但是存在另一个拉丁方:
     \begin{align*}
      \left[\begin{matrix}
          1&2&3&4&5\\
          4&5&1&2&3\\
          2&3&4&5&1\\
          5&1&2&3&4\\
          3&4&5&1&2\\
         \end{matrix}\right]
      \end{align*}
      与上述3个5阶拉丁方均正交。即5阶正交拉丁方族的大小应当为4,上述3个个5阶拉丁方不构成
      正交拉丁方族。
   \end{enumerate}
\end{solution}