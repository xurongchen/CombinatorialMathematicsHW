\begin{proof}
    可表示的证明:
    采用数学归纳法。
    \begin{itemize}
        \item \textbf{初始:}
        若$n=1$,显然序列$a_1=1, a_2=a_3=\ldots=0$满足:
        $\sum_{i\ge 1} {a_i\ i!} = 1$,即$n=1$可以表达。
        \item \textbf{归纳:}
        假设$n=m$存在序列$a_1,a_2,a_3,\ldots$的表达。希望构造$n=m+1$的表示
        序列。设$t$是最小的整数,使得$a_t\neq i$,则构造$a'_1,a'_2,a'_3,\ldots,a'_t,a'_{t+1},\ldots$,
        使得对于$k<t$,$a'_k=0$,对于$k>t$,$a'_k=a_k$,而$a'_t=a_t+1$。
        此时:
        \begin{align*}
            \sum_{i\ge 1} {a_i\ i!}-\sum_{i\ge 1} {a'_i\ i!} &= \sum_{i\ge 1}{(a_i-a'_i)\ i!} \\
            &= \sum_{1\le i< t}{(i-0)\ i!} + (a_t - (a_t+1))\ t! \\&+ \sum_{i> t}{(a_i - a_i)\ i!}\\
            &= -1
        \end{align*}
        即$\sum_{i\ge 1} {a'_i\ i!}=\sum_{i\ge 1} {a_i\ i!}+1=m+1$.
        归纳得$a'_1,a'_2,a'_3,\ldots,a'_t,a'_{t+1},\ldots$可表达$m+1$。
    \end{itemize}

    唯一性证明:
    假设$n$存在$a_1,a_2,a_3,\ldots$和$a'_1,a'_2,a'_3,\ldots$两种表达,由于两者不完全相同,
    则设$t$是最大的数使得$a_k\neq a'_k$,不是一般性地假设$a_k> a'_k$。则:
    \begin{align*}
        \sum_{i\ge 1} {a_i\ i!}-\sum_{i\ge 1} {a'_i\ i!} &= \sum_{i\ge 1}{(a_i-a'_i)\ i!} \\
        &= (a_k-a'_k)\ k! + \sum_{1\le i< k}{(a_i - a'_i)\ i!}\\
        &\ge k! - \sum_{1\le i< k}{a'_i\ i!}\\
        &\ge k! - \sum_{1\le i< k}{i\ i!}\\
        &\ge 1
    \end{align*}
    即必然有$\sum_{i\ge 1} {a_i\ i!}\neq\sum_{i\ge 1} {a'_i\ i!}$,这与假设矛盾,唯一性得证。
\end{proof}