\begin{solution}
    考虑新增的第$n$个点,在已有$n-1$个点的基础上的影响:显然,新的点和已有的$n-1$个点必然产生$n-1$条新的弦。不妨从新的点逆时针
    开始对所有点进行标记$1,2,\ldots, n-1$。则标为$i$的点与新点构成的弦对应两段弧(不包括两段的点$i$与新增点)上分别各有$i-1$和
    $n-i-1$个点。故点$i$与新点构成的弦经过$(i-1)(n-i-1)$条已有的弦。若新弦未经过任何已有的弦,其分割数目增益为1;若新弦每多经过
    一条已有弦,则额外带来1的分割数目增益。设第$n$个点的分割数目为$F_n$,则:
    \begin{align*}
        F_n-F_{n-1} &= \sum_{i=1}^{n-1} \left[(i-1)(n-i-1)+1\right] \\
        &= \sum_{i=1}^{n-1} \left(-i^2+in+2-n\right)\\
        &= - \frac{(n-1) n (2n-1)}{6} + n*\frac{n (n-1)}{2} + (2-n)(n-1)\\
        &= \frac{n^3-6n^2+17n-12}{6}
    \end{align*}
    相应地:
    \begin{align*}
        F_{n-1}-F_{n-2} &= \frac{(n-1)^3-6(n-1)^2+17(n-1)-12}{6}\\
        &= \frac{n^3-9n^2+32n-36}{6}
    \end{align*}
    于是:
    \begin{align*}
        F_n-2F_{n-1}+F_{n-2} = \frac{3n^2-15n+24}{6} = \frac{n^2-5n+8}{2}
    \end{align*}
    相应地:
    \begin{align*}
        F_{n-1}-2F_{n-2}+F_{n-3} = \frac{(n-1)^2-5(n-1)+8}{2} = \frac{n^2-7n+14}{2}
    \end{align*}
    于是:
    \begin{align*}
        F_n-3F_{n-1}+3F_{n-2}-F_{n-3} = \frac{2n-6}{2} = n-3
    \end{align*}
    相应地:
    \begin{align*}
        F_{n-1}-3F_{n-2}+3F_{n-3}-F_{n-4} = \frac{2n-6}{2} = (n-1)-3=n-4
    \end{align*}
    于是:
    \begin{align*}
        F_n-4F_{n-1}+6F_{n-2}-4F_{n-3}+F_{n-4} = 1
    \end{align*}
    相应地:
    \begin{align*}
        F_{n-1}-4F_{n-2}+6F_{n-3}-4F_{n-4}+F_{n-5} = 1
    \end{align*}
    于是:
    \begin{align*}
        F_n-5F_{n-1}+10F_{n-2}-10F_{n-3}+5F_{n-4}-F_{n-5} = 0
    \end{align*}
    得到特征方程:$x^5-5x^4+10x^3-10x^2+5x-1=0$,解得:
    \begin{align*}
        x_1 = x_2 = x_3 = x_4 = x_5 = 1
    \end{align*}
    设特征方程:
    $F_n = \alpha n^4+\beta n^3 +\gamma n^2 + \delta n +\epsilon$。代入初始值$F_1=1$,$F_2=2$,$F_3=4$,$F_4=8$,$F_5=16$:
    \begin{align*}
        \begin{cases}
            \alpha + \beta + \gamma + \delta +\epsilon= 1\\
            16\alpha + 8\beta + 4\gamma + 2\delta +\epsilon= 2\\
            81\alpha + 27\beta + 9\gamma + 3\delta +\epsilon= 4\\
            256\alpha + 64\beta + 16\gamma + 4\delta +\epsilon= 8\\
            625\alpha + 125\beta + 25\gamma + 5\delta +\epsilon= 16\\
        \end{cases}
    \end{align*}
    解得:
    % By oeis.org A000127
    % Maximal number of regions obtained by joining n points around a circle by straight lines. 
    % Also number of regions in 4-space formed by n-1 hyperplanes.
    % (Formerly M1119 N0427)
    % a(n) = (n^4 - 6*n^3 + 23*n^2 - 18*n + 24)/24.
    \begin{align*}
        \begin{cases}
            \alpha = \frac{1}{24}\\
            \beta = -\frac{1}{4}\\
            \gamma = \frac{23}{24}\\
            \delta = -\frac{3}{4}\\
            \epsilon = 1
        \end{cases}
    \end{align*}
    即:
    $$F_n = \frac{1}{24}\cdot n^4 - \frac{1}{4}\cdot n^3 + \frac{23}{24}\cdot n^2 - \frac{3}{4}\cdot n + 1$$
\end{solution}