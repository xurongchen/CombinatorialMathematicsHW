\begin{solution}
   如下图若所示的正六面体abcdefgh中,考虑空间旋转导致的顶点置换:
   \begin{center}
      \begin{tikzpicture}
         \node (O) at (0,0,0) {\small e};
         \node (A) at (0,\Width,0) {\small a};
         \node (B) at (0,\Width,\Height) {\small d};
         \node (C) at (0,0,\Height) {\small h};
         \node (D) at (\Depth,0,0) {\small f};
         \node (E) at (\Depth,\Width,0) {\small b};
         \node (F) at (\Depth,\Width,\Height) {\small c};
         \node (G) at (\Depth,0,\Height) {\small g};
         \draw (O) -- (C) -- (G) -- (D) -- (O);% Bottom Face
         \draw (O) -- (A) -- (E) -- (D) -- (O);% Back Face
         \draw (O) -- (A) -- (B) -- (C) -- (O);% Left Face
         \draw (D) -- (E) -- (F) -- (G) -- (D);% Right Face
         \draw (C) -- (B) -- (F) -- (G) -- (C);% Front Face
         \draw (A) -- (B) -- (F) -- (E) -- (A);% Top Face
   
         \draw[blue,dashed] (A) -- (G);
         \draw[brown,dashed] (0,\Width/2,\Height/2) -- (\Depth,\Width/2,\Height/2);
         \draw[red,dashed] (\Depth/2,0,\Height) -- (\Depth/2,\Width,0);
         \end{tikzpicture}
   \end{center}
   \begin{enumerate}
      \item 不变:
      \begin{align*}
         \left(
            \begin{matrix}
               a & b & c & d & e & f & g & h\\
               a & b & c & d & e & f & g & h
            \end{matrix}
            \right) &= (a)(b)(c)(d)(e)(f)(g)(h) \\
            &= (8,0,0,0,0,0,0,0)
      \end{align*}
      $1$种情况。
      \item 面心-面心的旋转(棕色线所示):
      \begin{itemize}
         \item 逆时针旋转90度或270度,例如置换:
         \begin{align*}
            \left(
               \begin{matrix}
                  a & b & c & d & e & f & g & h\\
                  e & f & b & a & h & g & c & d
               \end{matrix}
               \right) &= (aehd)(bfgc) \\
               &= (0,0,0,2,0,0,0,0)
         \end{align*}
         3个面各2种旋转,共$3*2 = 6$种情况。
      \item 旋转180度,例如置换:
      \begin{align*}
         \left(
            \begin{matrix}
               a & b & c & d & e & f & g & h\\
               h & g & f & e & d & c & b & a
            \end{matrix}
            \right) &= (ah)(bg)(cf)(de) \\
            &= (0,4,0,0,0,0,0,0)
      \end{align*}
      3个面,共$3$种情况。
      \end{itemize}
      \item 棱中点-棱中点的180度旋转(红色线所示),例如置换:
      \begin{align*}
         \left(
            \begin{matrix}
               a & b & c & d & e & f & g & h\\
               b & a & e & f & c & d & h & g
            \end{matrix}
            \right) &= (ab)(ce)(df)(gh) \\
            &= (0,4,0,0,0,0,0,0)
      \end{align*}
      6对棱,共$6$种情况。
      \item 顶点-顶点的逆时针120度或240旋转(蓝色线所示),例如置换:
      \begin{align*}
         \left(
            \begin{matrix}
               a & b & c & d & e & f & g & h\\
               a & d & h & e & b & c & g & f
            \end{matrix}
            \right) &= (a)(bde)(chf)(g) \\
            &= (2,0,2,0,0,0,0,0)
      \end{align*}
      4对顶点各2种旋转,共$4*2=8$种情况。
   \end{enumerate}
   合计$1+6+3+6+8=24$种情况。设未放置球的顶点为e(mpty)。由Polya定理,方案数的母函数$F$有:
   \begin{align*}
      F&=\frac{1}{24}[(b+r+e)^8+6(b^4+r^4+e^4)^2+9(b^2+r^2+e^2)^4\\
      &+8(b+r+e)^2(b^3+r^3+e^3)^2])\\
   \end{align*}
   其中$b^2r^2e^4$的系数为:
   \begin{align*}
      \frac{1}{24}\left(C_8^2C_6^2C_4^4+0+9C_4^1C_3^1C_2^2+0\right) = 22
   \end{align*}
   即两个r色球,两个b色的球装在六面体顶点的方案数为22种。
\end{solution}