\begin{solution}
   如下图若所示的正六面体abcdefgh中,考虑空间旋转导致的面置换.
   简记面abcd为面1;面dcgh为面2;面cbfg为面3;面fehg为面4;
   面adhc为面4;面baef为面6:
   \begin{center}
      \begin{tikzpicture}
         \node (O) at (0,0,0) {\small e};
         \node (A) at (0,\Width,0) {\small a};
         \node (B) at (0,\Width,\Height) {\small d};
         \node (C) at (0,0,\Height) {\small h};
         \node (D) at (\Depth,0,0) {\small f};
         \node (E) at (\Depth,\Width,0) {\small b};
         \node (F) at (\Depth,\Width,\Height) {\small c};
         \node (G) at (\Depth,0,\Height) {\small g};
         \draw (O) -- (C) -- (G) -- (D) -- (O);% Bottom Face
         \draw (O) -- (A) -- (E) -- (D) -- (O);% Back Face
         \draw (O) -- (A) -- (B) -- (C) -- (O);% Left Face
         \draw (D) -- (E) -- (F) -- (G) -- (D);% Right Face
         \draw (C) -- (B) -- (F) -- (G) -- (C);% Front Face
         \draw (A) -- (B) -- (F) -- (E) -- (A);% Top Face
   
         \draw[blue,dashed] (A) -- (G);
         \draw[brown,dashed] (0,\Width/2,\Height/2) -- (\Depth,\Width/2,\Height/2);
         \draw[red,dashed] (\Depth/2,0,\Height) -- (\Depth/2,\Width,0);
         \end{tikzpicture}
   \end{center}
   \begin{enumerate}
      \item 不变:
      \begin{align*}
         \left(
            \begin{matrix}
               1 & 2 & 3 & 4 & 5 & 6\\
               1 & 2 & 3 & 4 & 5 & 6
            \end{matrix}
            \right) &= (1)(2)(3)(4)(5)(6) \\
            &= (6,0,0,0,0,0)
      \end{align*}
      $1$种情况。
      \item 面心-面心的旋转(棕色线所示):
      \begin{itemize}
         \item 逆时针旋转90度或270度,例如置换:
         \begin{align*}
            \left(
               \begin{matrix}
                  1 & 2 & 3 & 4 & 5 & 6\\
                  2 & 4 & 3 & 6 & 5 & 1
               \end{matrix}
               \right) &= (1246)(3)(5) \\
               &= (2,0,0,1,0,0)
         \end{align*}
         3个面各2种旋转,共$3*2 = 6$种情况。
      \item 旋转180度,例如置换:
      \begin{align*}
         \left(
            \begin{matrix}
               1 & 2 & 3 & 4 & 5 & 6\\
               4 & 6 & 3 & 1 & 5 & 2
            \end{matrix}
            \right) &= (14)(26)(3)(5) \\
            &= (2,2,0,0,0,0)
      \end{align*}
      3个面,共$3$种情况。
      \end{itemize}
      \item 棱中点-棱中点的180度旋转(红色线所示),例如置换:
      \begin{align*}
         \left(
            \begin{matrix}
               1 & 2 & 3 & 4 & 5 & 6\\
               6 & 4 & 5 & 2 & 3 & 1
            \end{matrix}
            \right) &= (16)(24)(35) \\
            &= (0,3,0,0,0,0)
      \end{align*}
      6对棱,共$6$种情况。
      \item 顶点-顶点的逆时针120度或240旋转(蓝色线所示),例如置换:
      \begin{align*}
         \left(
            \begin{matrix}
               1 & 2 & 3 & 4 & 5 & 6\\
               6 & 3 & 4 & 2 & 1 & 5
            \end{matrix}
            \right) &= (165)(234) \\
            &= (0,0,2,0,0,0)
      \end{align*}
      4对顶点各2种旋转,共$4*2=8$种情况。
   \end{enumerate}
   合计$1+6+3+6+8=24$种情况。由Polya定理,染色方案的母函数$F$有:
   \begin{align*}
      F&=\frac{1}{24}[(g+r+b+y)^6+6(g^4+r^4+b^4+y^4)(g+r+b+y)^2\\
      &+3(g^2+r^2+b^2+y^2)^2(g+r+b+y)^2+6(g^3+r^3+b^3+y^3)^2\\
      &+8(g^2+r^2+b^2+y^2)^3]\\
   \end{align*}
   其中$g^2y^2br$的系数为:
   \begin{align*}
      \frac{1}{24}\left(C_6^2C_4^2C_2^1C_1^1+0+3C_2^1C_1^1C_2^1C_1^1+0+0\right) = 8
   \end{align*}
   即两个面用色g,两个面用色y,其余一面用b,一面用r的方案有8种。
\end{solution}